% This is samplepaper.tex, a sample chapter demonstrating the
% LLNCS macro package for Springer Computer Science proceedings;
% Version 2.21 of 2022/01/12
%
\documentclass[runningheads]{llncs}
%
\usepackage[T1]{fontenc}
% T1 fonts will be used to generate the final print and online PDFs,
% so please use T1 fonts in your manuscript whenever possible.
% Other font encondings may result in incorrect characters.
%
\usepackage{graphicx}
\usepackage{hyperref}
% Used for displaying a sample figure. If possible, figure files should
% be included in EPS format.
%
% If you use the hyperref package, please uncomment the following two lines
% to display URLs in blue roman font according to Springer's eBook style:
\usepackage{color}
\renewcommand\UrlFont{\color{blue}\rmfamily}
\urlstyle{rm}

\usepackage{amsmath}
\usepackage{amssymb}
\usepackage{tikz}
\usepackage{wrapfig}
\usepackage{environ}

\usepackage{changes}
\usepackage{amstex}
\usepackage{algorithm}
\usepackage{algpseudocode}
\usepackage{varwidth}
\setauthormarkup{}
\definechangesauthor[name=jo, color=magenta]{JO}
\definechangesauthor[name=rom, color=teal]{ROM}

% set of real numbers
\newcommand{\R}{\ensuremath{\mathbb{R}}}
% set of integer numbers
\newcommand{\Z}{\ensuremath{\mathbb{Z}}}
% cell complex build on the lattice \Z^d
\let\C\relax \DeclareMathOperator{\C}{\ensuremath{\mathcal{C}^d}} % modified because of incompatiblity with hyperref package
% cardinality of a set
\newcommand{\Card}[1]{\ensuremath{\#\left( #1 \right)}}
% Interior of a closed set.
\newcommand{\Int}[1]{\ensuremath{\mathrm{Int}\left( #1 \right)}}
% Some vectors and matrices
\newcommand{\vx}[0]{\ensuremath{\mathbf{x}}}
\newcommand{\vy}[0]{\ensuremath{\mathbf{y}}}
\newcommand{\vz}[0]{\ensuremath{\mathbf{z}}}
\newcommand{\vb}[0]{\ensuremath{\mathbf{b}}}
\newcommand{\vA}[0]{\ensuremath{\mathbf{A}}}
\newcommand{\va}[0]{\ensuremath{\mathbf{a}}}
\newcommand{\vc}[0]{\ensuremath{\mathbf{c}}}
\newcommand{\ve}[0]{\ensuremath{\mathbf{e}}}
\newcommand{\vm}[0]{\ensuremath{\mathbf{m}}}
\newcommand{\vn}[0]{\ensuremath{\mathbf{n}}}
\newcommand{\vu}[0]{\ensuremath{\mathbf{u}}}
\newcommand{\vv}[0]{\ensuremath{\mathbf{v}}}
\newcommand{\vw}[0]{\ensuremath{\mathbf{w}}}
\newcommand{\vp}[0]{\ensuremath{\mathbf{p}}}
\newcommand{\vq}[0]{\ensuremath{\mathbf{q}}}
\newcommand{\vr}[0]{\ensuremath{\mathbf{r}}}
\newcommand{\vs}[0]{\ensuremath{\mathbf{s}}}
\newcommand{\vt}[0]{\ensuremath{\mathbf{t}}}
% Convex hull
\newcommand{\Conv}[1]{\mathrm{CvxH}\left( #1 \right)}

% References
\newcommand{\Equ}[1]{(\ref{#1})}
\newcommand{\RefFigure}[1]{Figure~\ref{#1}}
\newcommand{\RefLemma}[1]{Lemma~\ref{#1}}
\newcommand{\RefLemmas}[2]{Lemmas~\ref{#1} and~\ref{#2}}
\newcommand{\RefTheorem}[1]{Theorem~\ref{#1}}
\newcommand{\RefCorollary}[1]{Corollary~\ref{#1}}
\newcommand{\RefAlgorithm}[1]{Algorithm~\ref{#1}}
\newcommand{\RefSection}[1]{Section~\ref{#1}}
\newcommand{\Refline}[1]{line~\ref{#1}}
\newcommand{\RefLine}[1]{Line~\ref{#1}}

% Star, closure, boundary of a complex
\newcommand{\Bd}[1]{\ensuremath{\mathrm{Bd}\left(#1\right)}}
\newcommand{\Star}[1]{\ensuremath{\mathrm{Star}\left(#1\right)}}
\newcommand{\Starop}[0]{\ensuremath{\mathrm{Star}}}
\newcommand{\Cl}[1]{\ensuremath{\mathrm{Cl}\left(#1\right)}}
% Closed interior
\newcommand{\CI}[1]{\ensuremath{\mathrm{ClInt}\left(#1\right)}}
% Body of a complex
\newcommand{\Body}[1]{\ensuremath{\left\|#1\right\|}}
% topological closure
\newcommand{\TCl}[1]{\ensuremath{\overline{#1}}}
% topological interior
\newcommand{\TInt}[1]{\ensuremath{\mathring{#1}}}
% topological boundary
\newcommand{\TBd}[1]{\ensuremath{\partial{#1}}}
% <=
\newcommand{\Le}{\ensuremath{\leqslant}}
% >=
\newcommand{\Ge}{\ensuremath{\geqslant}}


% Euclidean distance
\newcommand{\EDist}[0]{\ensuremath{\mathrm{d}_\mathbf{E}}}
% Hausdorff distance
\newcommand{\HDist}[0]{\ensuremath{\mathrm{d}_H}}
% Normal cone to S at p
\newcommand{\NC}[2]{\ensuremath{\mathrm{N}_{#1}(#2)}}

% l2 norm
\newcommand{\ltwo}[0]{\ensuremath{\ell_2}}
\newcommand{\loo}[0]{\ensuremath{\ell_\infty}}

% Reach
\newcommand{\Reach}[1]{\ensuremath{\mathrm{reach}(#1)}}
% Volume
\newcommand{\Vol}[1]{\ensuremath{\mathrm{Vol}\left(#1\right)}}
\newcommand{\Vold}[2]{\ensuremath{\mathrm{Vol}^{#1}\left(#2\right)}}
% Digitization
\newcommand{\Dig}[2]{\ensuremath{\mathrm{D}_{#1}\left(#2\right)}}
% Counting lattice points
\newcommand{\Lat}[1]{\ensuremath{\mathcal{L}_{#1}}}
\newcommand{\LatC}[2]{\ensuremath{\Lat{#1}(#2)}}


%
\begin{document}
%
\title{Fast and exact visibility on digitized shapes and application to feature-aware normal estimation}
%
    \titlerunning{Fast and exact visibility on digitized shapes}
% If the paper title is too long for the running head, you can set
% an abbreviated paper title here
%
    \author{Romain Negro\inst{1}\orcidID{0000-XXXX-YYYY-ZZZZ} \and
    Jacques-Olivier Lachaud\inst{1}\orcidID{0000-0003-4236-2133}}
%
    \authorrunning{R. Negro and J.-O. Lachaud}
% First names are abbreviated in the running head.
% If there are more than two authors, 'et al.' is used.
%
    \institute{Universit\'e Savoie Mont Blanc, CNRS, LAMA, F-73000 Chambéry, France\\
    \email{\{romain.negro|jacques-olivier.lachaud\}@univ-smb.fr}}
%
    \maketitle              % typeset the header of the contribution
%
    \begin{abstract}
        Our abstract
        \keywords{Visibility \and Geometric inference \and Digital normal estimation \and Digital geometry}
    \end{abstract}

%%%%%%%%%%%%%%%%%%%%%%%%%%%%%%%%%%%%%%%%%%%%%%%%%%%%%%%%%%%%%%%%%%%%%%


    \section{Introduction}


    Definition Zd
    Definition Cell complex
    Definition Star

%%%%%%%%%%%%%%%%%%%%%%%%%%%%%%%%%%%%%%%%%%%%%%%%%%%%%%%%%%%%%%%%%%%%%%


    \section{Visibility through tangency of chords}

    Introduce full convexity and tangency. \cite{lachaud:2021-dgmm,lachaud:2022-jmiv}.

    \begin{definition}
        Two points \( p \) and \( q \) on \( K \) are visible if and only if the segment \([p, q]\) is included in \(\text{Star}(K)\), i.e., \( [p, q] \subseteq \text{Star}(K)\).
    \end{definition}

    % Examples of visibility in 2D
    \begin{figure}
        \centering
        \begin{tikzpicture}
            \draw[dashed] (0,0) -- (4,2);
            \filldraw[black] (0,0) circle (0.05) node[anchor=north] {p1};
            \filldraw[black] (4,2) circle (0.05) node[anchor=north] {q1};
            \draw (0,0) -- (1,0) -- (2,0) -- (2,1) -- (3,1) -- (3,2) -- (4,2);
            \draw[dashed=gray] (-1,0) -- (-1,1) -- (0,1) -- (1,1) -- (1,2) -- (2,2) -- (2,3) -- (3,3) -- (4,3) -- (5,3) -- (5,2) -- (5,1) -- (4,1) -- (4,0) -- (3,0) -- (3,-1) -- (2,-1) -- (1,-1) -- (0,-1) -- (-1,-1) -- (-1,0);
        \end{tikzpicture}
        \hspace{\floatsep}
        \begin{tikzpicture}
            \draw[red,dashed] (0,0) -- (3,2);
            \filldraw[black] (0,0) circle (0.05) node[anchor=north] {p2};
            \filldraw[black] (3,2) circle (0.05) node[anchor=north] {q2};
            \draw (0,0) -- (0,1) -- (0,2) -- (1,2) -- (1,3) -- (2,3) -- (2,2) -- (3,2);
            \draw[dashed=gray] (-1,0) -- (-1,1) -- (-1,2) -- (-1,3) -- (0,3) -- (0,4) -- (1,4) -- (2,4) -- (3,4) -- (3,3) -- (4,3) -- (4,2) -- (4,1) -- (3,1) -- (2,1) -- (1,1) -- (1,0) -- (1,-1) -- (0,-1) -- (-1,-1) -- (-1,0);
        \end{tikzpicture}
        \newline
        \newline
        \begin{tikzpicture}
            \draw[red,dashed] (0,0) -- (2,0);
            \filldraw[black] (0,0) circle (0.05) node[anchor=north] {p3};
            \filldraw[black] (2,0) circle (0.05) node[anchor=north] {q3};
            \draw (0,0) -- (0,1) -- (0,2) -- (1,2) -- (2,2) -- (2,1) -- (2,0);
            \draw[dashed=gray] (-1,-1) -- (-1,3) -- (3,3) -- (3,-1) -- (1,-1) -- (1,1) -- (1,-1) -- (-1,-1);
        \end{tikzpicture}
        \hspace{\floatsep}
        \begin{tikzpicture}
            \draw[red,dashed] (0,0) -- (2,2);
            \filldraw[black] (0,0) circle (0.05) node[anchor=north] {p4};
            \filldraw[black] (2,2) circle (0.05) node[anchor=north] {q4};
            \draw (0,0) -- (0,1) -- (0,2) -- (1,2) -- (2,2);
            \draw[dashed=gray] (-1,0) -- (-1,1) -- (-1,2) -- (-1,3) -- (0,3) -- (1,3) -- (2,3) -- (3,3) -- (3,2) -- (3,1) -- (2,1) -- (1,1) -- (1,0) -- (1,-1) -- (0,-1) -- (-1,-1) -- (-1,0);
        \end{tikzpicture}
        \caption{Examples of visibility in 2D, only q1 is visible from p1, q2 is not visible from p2 (gets out of the star), q3 is not visible from p3 (gets out of the star from a 1-d cell), q4 is not visible from p4 (gets out of the star from a 0-d cell).}
        \label{fig:visibility-2d}
    \end{figure}

    % Example of visibility not necessarily connected
    % Path from p to q : r u r u r r r r r u r
    \begin{figure}
        \centering
        \begin{tikzpicture}
            \draw (0,0) -- (1,0) -- (1,1) -- (2,1) -- (2,2) -- (3,2) -- (4,2) -- (5,2) -- (6,2) -- (7,2) -- (7,3);
            \draw [black, dashed] (0,0) -- (7,3);
            \draw [red, dashed] (0,0) -- (7,2);
            \draw [red, dashed] (0,0) -- (6,2);
            \filldraw[black] (0,0) circle (0.05) node[anchor=north] {p};
            \filldraw[gray] (1,0) circle (0.05);
            \filldraw[gray] (1,1) circle (0.05);
            \filldraw[gray] (2,1) circle (0.05);
            \filldraw[gray] (2,2) circle (0.05);
            \filldraw[gray] (3,2) circle (0.05);
            \filldraw[gray] (4,2) circle (0.05);
            \filldraw[gray] (5,2) circle (0.05);
            \filldraw[red] (6,2) circle (0.05);
            \filldraw[red] (7,2) circle (0.05);
            \filldraw[black] (7,3) circle (0.05) node[anchor=west] {q};
            \draw[dashed=gray] (-1,-1) -- (-1,1) -- (0,1) -- (0,2) -- (1,2) -- (1,3) -- (6,3) -- (6,4) -- (8,4) -- (8,1) -- (3,1) -- (3,0) -- (2,0) -- (2,-1) -- (-1,-1);
        \end{tikzpicture}
        \caption{Example of non connex visibility in 2D, q is visible from p while the visibility points are not 26-connected.}
        \label{fig:visibility-2d-not-connected}
    \end{figure}


%%%%%%%%%%%%%%%%%%%%%%%%%%%%%%%%%%%%%%%%%%%%%%%%%%%%%%%%%%%%%%%%%%%%%%


    \section{Fast computation using integer intervals intersections}

    \begin{algorithm}
        \caption{Given a cell complex C and a radius \(r\), compute the visibility at every point of C up to distance \(r\).}
        \label{alg:visibility}
        \begin{algorithmic}

            \Function{CheckI}{$I$: Interval, ComplexI: list of Interval}
                \State $[a, b] \gets I$
                \State $Result \gets \emptyset$
                \For{each interval $J$ in ComplexI}
                    \State $[c, d] \gets J$
                    \If{$\text{Size}(I) \leq \text{Size}(J)$}
                        \State $Result \gets Result \cup [c-a,d-b+1]$
                    \EndIf
                \EndFor
                \State \Return Result
            \EndFunction

            \Function{MatchVector}{\text{CurrentIResult}: list of Interval, VectorI: list of Interval, ComplexI: list of Interval}
                \For{each interval $I$ in VectorI}
                    \State $\text{CurrentIResult} \gets \text{CurrentIResult} \cap \Call{CheckI}{I, \text{ComplexI}}$
                \EndFor
                \State \Return \text{CurrentIResult}
            \EndFunction

            \Function{Visibility}{\text{C}: Cell complex, \text{Radius}: integer}
                \State $FigLattices \gets \Call{GetStarFigLattices}{C}$
                \State $Vectors \gets \Call{GetAllPrimalVectors}{Radius}$
                \State $Visibility: \text{list of boolean} \gets \emptyset$
                \State $minFInterval, maxFInterval \gets \Call{GetMinMaxInterval}{FigLattices}$
                \For{each $V$ in $Vectors$}
                    \State $VLattice \gets \Call{GetLattice}{\text{V}}$
                    \For{every shift $tx, ty$ in $FigLattices$}
                        \State $CurrentIResult \gets [minFInterval, maxFInterval]$
                        \State $StartingPoint \gets \Call{GetStartingPoint}{tx, ty}$
                        \For{each $Lattice$ in $VStarLattice$}
                            \State \begin{varwidth}[t]{\linewidth}
                                       $CurrentIResult~\gets~MatchVector(CurrentIResult,$\par
                                       \hskip\algorithmicindent $Lattice.intervals,$\par
                                       \hskip\algorithmicindent $\text{FigLattices}[StartingPoint + Lattice.position])$
                            \end{varwidth}
                        \EndFor
                        \State \Call{UpdateVisibility}{$Visibility$, $CurrentIResult$}
                    \EndFor
                \EndFor
                \State \Return $Visibility$
            \EndFunction



        \end{algorithmic}
    \end{algorithm}


%%%%%%%%%%%%%%%%%%%%%%%%%%%%%%%%%%%%%%%%%%%%%%%%%%%%%%%%%%%%%%%%%%%%%%


    \section{Feature-aware normal estimation on digital surfaces}

%%%%%%%%%%%%%%%%%%%%%%%%%%%%%%%%%%%%%%%%%%%%%%%%%%%%%%%%%%%%%%%%%%%%%%


    \section{Conclusion}


%%%%%%%%%%%%%%%%%%%%%%%%%%%%%%%%%%%%%%%%%%%%%%%%%%%%%%%%%%%%%%%%%%%%%%
    \begin{credits}
        \subsubsection{\ackname}
        This work is partially supported by the French National Research Agency
        within the StableProxies project (ANR-22-CE46-0006).
%% \subsubsection{\discintname}
%% It is now necessary to declare any competing interests or to specifically
%% state that the authors have no competing interests. Please place the
%% statement with a bold run-in heading in small font size beneath the
%% (optional) acknowledgments\footnote{If EquinOCS, our proceedings submission
%% system, is used, then the disclaimer can be provided directly in the system.},
%% for example: The authors have no competing interests to declare that are
%% relevant to the content of this article. Or: Author A has received research
%% grants from Company W. Author B has received a speaker honorarium from
%% Company X and owns stock in Company Y. Author C is a member of committee Z.
    \end{credits}
%
% ---- Bibliography ----
%
% BibTeX users should specify bibliography style 'splncs04'.
% References will then be sorted and formatted in the correct style.
%
    \bibliographystyle{splncs04}
    \bibliography{biblio}

\end{document}
