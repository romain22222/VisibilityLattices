\documentclass[10pt,french]{letter}
%\usepackage{}
\usepackage[T1]{fontenc}
\usepackage[latin1]{inputenc}
\usepackage[french]{babel}
\usepackage{a4}
\usepackage{epsfig}
\usepackage{amssymb}
\usepackage{amsmath}

\usepackage{fancyhdr}
\usepackage{hyperref}
\usepackage[left=3cm,right=3cm,top=3cm,bottom=3cm]{geometry}

%\geometry{hmargin=2.5cm,vmargin=3cm}

%% \setlength{\parindent}{0.5cm}
%% \setlength{\topmargin}{-4cm}
%\setlength{\textwidth}{15cm}
%% \setlength{\oddsidemargin}{0.5cm}
%% \setlength{\parindent}{0.5cm}

\pagestyle{fancy}
%\headheight 35pt


%% \lhead{\noindent\begin{minipage}{0.3\textwidth}
%%         %  \epsfig{width=\textwidth,file=logo-us.eps}\\
%%   		\end{minipage}
%%   \hfill
%%   \begin{minipage}{0.4\textwidth}
%%     Laboratoire de Math�matiques\\
%%     UMR CNRS 5127\\
%%     UFR SceM, Campus scientifique \\
%%     73376 Le-Bourget-du-Lac Cedex\\
%%   \end{minipage}
%% }

\cfoot{\tiny Laboratoire de Math�matiques, UMR CNRS 5127 --- UFR SceM, campus scientifique --- 73376 Le-Bourget-du-Lac Cedex }

\renewcommand{\headrulewidth}{0.4pt}
\renewcommand{\footrulewidth}{0.4pt}


\def\NAME{Jacques-Olivier Lachaud}
\address{\NAME \\
  Laboratoire de Math�matiques, UMR CNRS 5127\\
  UFR SceM, Campus scientifique \\
  73376 Le-Bourget-du-Lac Cedex\\
}
\signature{\NAME}
\date{Le-Bourget-du-Lac, le \today}

\pagestyle{fancy}

\begin{document}

%% %  \underline{\it Objet :} Lettre de recommandation pour M Xavier Proven�al
%% \mbox{~}
%% \vspace{-0.5cm}

%% \begin{center}
%%   {\large \bf  Lettre de recommandation de M. Romain \textsc{Negro} pour un stage au CERN dans le cadre de son cursus CMI Informatique}

%% \end{center}
%% \bigskip

\mbox{~}\hfill\mbox{Le-Bourget-du-Lac, February 4, 2026}
\medskip

Dear editors

Please find enclosed our paper

``Controlled peeling of fully convex digital sets and thin envelope''

which we submit to the JMIV special issue on Discrete Geometry and
Mathematical Morphology. It is the extension of our DGMM2025 paper
"How to peel fully convex sets".

As announced in our expression of interest message, we have extended
the paper along the following lines:
\begin{itemize}
\item We have proven that, given $Y$ and $X$ fully convex sets, such
  that $Y$ is subset of $X$, one can always peel $X$ without touching
  $Y$. In other words, in the inclusion hierarchy of fully convex
  sets, there is always a path through peeling between a set and its subsets.
\item With this property, we were able to design a new fully convex
  operator, called thin envelope, using dilation then parallel and
  sequential peeling. This new operator offers much more geometric
  guarantees than the previous ones, especially it stays at tight
  bounded distance from the Euclidean convex hull.
\item We have illustrated the use of the thin envelope to build
  digital polyhedral models. The resulting digital sets are thinner
  than the ones proposed in the literature, with a particularly
  faithful wireframe representation.
\item We have added some illustrations to help the reading.
\item The paper is now 17 pages long in 2-column format.
\end{itemize}

We believe this extended version of our paper will be a valuable
contribution to the JMIV special issue, and we look forward to hearing
from you soon.

Best regards
\bigskip

\mbox{~}
\hfill
\begin{minipage}{0.32\textwidth}
  Fabien Feschet \\
  Professor of Computer Science\\
  University Clermont Auvergne\\
  LIMOS, CNRS, ENSMSE
\end{minipage}
\hfill
\begin{minipage}{0.32\textwidth}
  Jacques-Olivier Lachaud\\
  Professor of Computer Science\\
  University Savoie Mont Blanc\\
  CNRS, LAMA
\end{minipage}
\hfill
\mbox{~}
\end{document}

