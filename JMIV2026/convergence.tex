
\newcommand{\CBh}[2]{\ensuremath{\partial_{#1} {#2}}}
\newcommand{\DBh}[2]{\ensuremath{\Delta_{#1} {#2}}}
\newcommand{\DTwo}[2]{\ensuremath{D_{2}\left( #1, #2 \right)}}
\newcommand{\DInf}[2]{\ensuremath{D_{\infty}\left( #1, #2 \right)}}

\noindent\textbf{Introduction, useful notations and properties.}
The visibility normal estimator can be defined for different sizes $k$
of stars. Therefore, for such a given size $k$, we denote by
$\vec{n}^k_V(p)$ the \emph{$k$-visibility normal at point $p$}.
The objective of this section is to demonstrate under which conditions
the $k$-visibility normal estimation is a multigrid convergent normal
estimator.

We restrict our study to the digitizations of compact subsets $X$ of
$\R^d$, whose topological boundary $\partial X$ has positive reach
greater than $\rho > 0$.  Let $h > 0$ be the digitization grid step
and let $X_h := X \cap \Z^d$ be the Gauss digitization of $X$ with
step $h$. Seeing the set $X_h$ as the union of hypercubes with edge
length $h$ and centered on points of $X_h$, we call its topological
boundary the {\em digitized boundary} $\CBh{h}{X}$ of $X$ of step
$h$. The intersection of $\CBh{h}{X}$ with $(h(\Z+\frac{1}{2}))^d$ is
denoted by $\DBh{h}{X}$ and its elements are called the {\em digital
  pointels} of $\CBh{h}{X}$. They correspond to the digital sets $Z$
of the previous sections, and we study the visibility between pointels
of $\DBh{h}{X}$ within its $k$-Star.


Finally let $\xi$ be the orthogonal projector of $\R^d$ onto $\partial
X$, which is well defined for any point within the reach of $\partial
X$. In fact we only require points to be locally within the reach of
$\partial X$, so staying at distance below the \emph{local feature
size} (see \cite{amenta:2001-cgta}) is enough, but would require more
care in the theorem formulation and proofs. The following lemma will be useful.

\begin{lemma}{\cite{Lachaud:2016-jmiv}}
  \label{lem-hausdorff-close}
  Let $h$ be smaller than $\frac{2\rho}{\sqrt{d}}$.
  Let $p \in \CBh{h}{X}$. Then the point $p' := \xi(p)$ is at
  Euclidean distance from $p$ lesser or equal to
  $\frac{\sqrt{d}}{2}h$. Moreover $p'$ is the point of $\partial X$
  that is closest to $p$, and $(pp')$ is aligned with the normal of
  $\partial X$ at $p'$.
\end{lemma}

\noindent\textbf{Geometric properties on digitized boundary.}
The following theorem shows that the vector joining distant enough
points of the digitized boundary tend to be orthogonal to the smooth
boundary of $X$.
\begin{theorem}
  Let $h$ the gridstep be a positive number smaller than $h_0 :=
  \frac{2\rho}{\sqrt{d}}$. Let $p, q$ be two pointels of
  $\DBh{h}{X}$. It holds that:
  \begin{align}
    \label{eq-pq-dot-n-upper-bound}
    |\vec{pq} \cdot \vn| \Le \sqrt{d}h + \frac{(1+\sqrt{d})^2}{2}
    \frac{\|\vec{pq}\|^2}{\rho} + O\left(
    \frac{\|\vec{pq}\|^4}{\rho^3} \right),
  \end{align}
  where $\vn$ is the normal of $\partial X$ at $\xi(q)$. 
\end{theorem}
\begin{proof}
  Since $h \le h_0$, both $p$ and $q$ are within the reach of
  $\partial X$ and have unique projection onto it. Let $p'=\xi(p)$,
  $q'=\xi(q)$. We consider the plane $P$ tangent to $\partial X$ at
  $q'$, with normal $\vec{n}$. Finally we denote by $p''$ the
  projection of $p'$ onto $P$. We have
  \begin{align*}
    | \vec{pq} \cdot \vn | & = | \vec{pp'} \cdot \vn  +  \vec{p'p''} \cdot \vn  + \vec{p''q'} \cdot \vn  +  \vec{q'q} \cdot \vn | \\
    & \quad \text{(By linearity of scalar product)}\\
    & \Le | \vec{pp'} \cdot \vn | + | \vec{p'p''} \cdot \vn | + |\vec{p''q'} \cdot \vn | + | \vec{q'q} \cdot \vn | \\
    & \quad \text{(by triangular inequality)} \\
    & \Le \| \vec{pp'} \| + \| \vec{p'p''} \| + 0 + \| \vec{q'q} \| \\
    & \quad \text{(since $\vec{p''q'}$ is parallel to $P$)}\\
    & \Le \sqrt{d}h + \| \vec{p'p''} \|. \\
    & \quad \text{(using \RefLemma{lem-hausdorff-close} twice)}
  \end{align*}
  Since $\partial X$ has positive reach greater than $\rho$, we know
  that $\partial X$ is sandwiched between two balls tangent to $P$ at
  $q'$. Their centers are $q' \pm \rho \vec{n}$. Let $c$ be the center
  of the ball that is on the same side of $P$ as $p'$, and let us
  denote $r$ the projection of $p'$ onto the line going through both
  ball centers (and $q$ and $q'$). From the sandwich property, $p'$
  cannot lie in any of these balls so $\| \vec{p'c} \| \Ge
  \rho$. Besides it holds that
  \begin{align*}
    \| \vec{p'c} \|^2 & = \| \vec{{p'}r} \|^2 + \| \vec{rc} \|^2 \\
    & \quad \text{(By Pythagoras theorem)} \\
    & = \| \vec{p''q'} \|^2 + \| \vec{rq'} +\vec{q'c} \|^2 \\
    & \quad \text{(by construction of $p''$ and $r$)} \\
    & = \| \vec{p''q'} \|^2 + ( \rho - \|\vec{p'p''}\|)^2. \\
    & \quad \text{(since $\vec{rq'}$ and $\vec{q'c}$ are aligned and $\vec{p'p''}=\vec{rq'}$)}
  \end{align*}
  With the lower bound $\| \vec{p'c} \| \Ge \rho$, it entails (with Taylor expansion):
  \begin{align*}
    \|\vec{p'p''}\| & \Le \rho - \sqrt{\rho^2 - \|\vec{p''q'} \|^2} \\
    & = \frac{1}{2}\frac{\|\vec{p''q'} \|^2}{\rho} + O \left( \frac{\|\vec{p''q'} \|^4}{\rho^3} \right).
  \end{align*}
  Now $\|\vec{p''q'} \| \Le \| \vec{p'q'} \| = \| \vec{p'p} + \vec{pq}
  + \vec{qq'} \| \Le \sqrt{d}h+\|\vec{pq}\| \Le
  (1+\sqrt{d})\|\vec{pq}\|$, the last inequality coming from the fact
  that $p \neq q$, so $1 \Le \frac{\|\vec{pq}\|}{h}$. We obtain:
  \begin{align*}
    \|\vec{p'p''}\| \Le \frac{(1+\sqrt{d})^2}{2}\frac{\|\vec{pq} \|^2}{\rho} + O \left( \frac{\|\vec{pq} \|^4}{\rho^3} \right).
  \end{align*}
  We conclude since we established above $| \vec{pq} \cdot \vn | \Le
  \sqrt{d}h + \| \vec{p'p''} \|$.
\end{proof}

In order that the line $(pq)$ be at most orthogonal to the normal of
the smooth boundary, points $p$ and $q$ should not be too close
(otherwise there is a strong uncertainty on their positions) nor too
far away (the normal vector may vary too much). This is explicited in
the following corollary, which is immediate by dividing
\Equ{eq-pq-dot-n-upper-bound} by $\|\vec{pq}\|$.
\begin{corollary}
  \label{cor1}
  For $p \neq q$, let $\alpha$ be the angle between $\vec{pq}$ and $\vec{n}$. Then we have:
  \begin{align*}
    |\cos \alpha|
    % = \frac{|\vec{pq} \cdot \vn|}{\|\vec{pq}\|}
    \Le  \sqrt{d} \frac{h}{\|\vec{pq}\|} + \frac{(1+\sqrt{d})^2}{2}
    \frac{\|\vec{pq}\|}{\rho} + O\left(
    \frac{\|\vec{pq}\|}{\rho} \right)^3 \!\!.
  \end{align*}
  To minimize this value, it is optimal to choose pairs of points $(p,q)$ such that $\|\vec{pq}\|=\Theta(\sqrt{h})$. In this case, we derive $|\cos \alpha| \Le \Theta(\sqrt{h}) + O(\sqrt{h})^3$. 
\end{corollary}

\noindent\textbf{Distance between visible points.}  This subsection
proves that the average distance between $k$-visible pointels is some
$\Theta(\sqrt{h})$, starting from $k\Ge 2$ for dimension 2 and 3.

\begin{lemma}\label{lem-segment-is-close-to-dX}
  Let $a,b$ be points of $\partial X$, where the straight line segment
  $\lbrack ab \rbrack$ has length $l$. If $l \Le 2\rho$ then $\max_{y
    \in \lbrack ab \rbrack} d(y, \partial X) \Le \frac{l^2}{4\rho}$,
  where $d(y,Z):=\inf_{z \in Z} \| y-z\|$.
\end{lemma}
\begin{proof}
  We recall that, since $\partial X$ has positive reach greater than
  $\rho$, it means that around point $a$, $\partial X$ has an empty
  intersection with the open ball of radius $\rho$ tangent to $a$
  (inside and outside). Therefore the distance $\max_{y \in \lbrack ab
    \rbrack} d(y, \partial X)$ cannot exceed the distance of a chord
  of length $l=\| \vec{ab}\|$ to a sphere of radius $\rho$, hence the
  circle of radius $\rho$ in the plane containing $a$, $b$ and the
  sphere center. By an abuse of notation, denoting $a$ and $b$ also
  the extremeties of this chord, $m$ its middle point, $c$ the center
  of the circle, and $p$ the closest point to $m$ on the circle, our
  objective is to determine the distance $x:=\| \vec{mp}\|$. Using the
  fact that $(cm)$ is orthogonal to $(ab)$, Pythagoras theorem implies
  $\|\vec{am}\|^2 + \|\vec{mc}\|^2 = \|\vec{ac}\|^2$. Substituting by
  $x$, $l$ and $\rho$ gives
  \begin{align*}
    \frac{l^2}{4}+(\rho-x)^2 = \rho^2 &
    \quad \Rightarrow \quad 4x^2 - 8 \rho x + l^2 = 0.
  \end{align*}
  Since $l \Le 2\rho$ by hypothesis, the solution to this equation is
  $x=\rho-\sqrt{\rho^2-\frac{l^2}{4}}$. To conclude, one can check
  easily that $x \Le \frac{l^2}{4\rho}$ if, and only if, $l \Le
  2\rho$.
\end{proof}

\begin{theorem}\label{thm-distance-pq-to-dX}
  Let $p,q$ be two pointels of $\DBh{h}{X}$. If $\| \vec{pq}\|\Le
  \sqrt{\alpha h} - \sqrt{d}h$, then the segment $\lbrack p q \rbrack$
  stays close to $\CBh{h}{X}$ and more precisely:
  \begin{align*}
    \max_{r \in \lbrack p, q \rbrack} d( r, \CBh{h}{X}) \Le (\sqrt{d}+\frac{\alpha}{4\rho})h.
  \end{align*}
\end{theorem}
\begin{proof}
  Let $a=\xi(p)$ and $b=\xi(p)$ the (unique) projections of $a$ and
  $b$ onto $\partial X$. Then, by triangular inequality,
  \begin{align*}
    \max_{r \in \lbrack p q \rbrack} d( r, \CBh{h}{X})
    \Le & \max_{r \in \lbrack p q \rbrack} d( r, \lbrack a b \rbrack )
     + \max_{s \in \lbrack a b \rbrack} d( s, \partial X ) \\
    & + \max_{x \in \partial X} d( x, \CBh{h}{X} ).
  \end{align*}
  For the first term, we can write $r=(1-\lambda)p+\lambda q$, for
  some $\lambda \in \lbrack 0,1 \rbrack$. Let $s=(1-\lambda)a+\lambda
  b$. It is clearly a point of segment $\lbrack a b \rbrack$. We have:
  \begin{align*}
    d(r,s) & = \| (1-\lambda)p+\lambda q - ((1-\lambda)a+\lambda b) \| \\
    & \Le (1-\lambda)\|p-a\| + \lambda \|q-b\| \Le \frac{\sqrt{d}}{2}h,
  \end{align*}
  where the last inequality derives from
  \RefLemma{lem-hausdorff-close}.

  For the second term,
  \RefLemma{lem-segment-is-close-to-dX} induces
  \begin{align*}
    \max_{s \in \lbrack a b \rbrack} d( s, \partial X ) &\Le \frac{\|\vec{ab}\|^2}{4\rho}.
  \end{align*}
  But $\|\vec{ab}\| \Le \|\vec{pq}\| + \|\vec{pa}\| + \|\vec{qb}\| \Le
  \| \vec{pq}\| + \sqrt{d}h \Le \sqrt{\alpha h}$, using
  \RefLemma{lem-hausdorff-close} and the hypothesis on the distance of
  $p$ to $q$. The second term is thus bounded by $\frac{\alpha
    h}{4\rho}$.

  The last term is bounded by $\frac{\sqrt{d}}{2}h$
  (\RefLemma{lem-hausdorff-close} again), and the conclusion follows.
\end{proof}

\begin{theorem}
  Assume $\alpha:=4\rho(k-\sqrt{d})$ is positive. Then, for any
  gridstep $h$, $0< h < \min(\frac{\alpha}{d}, 2\rho)$, there exists a
  constant $\beta >0 $ such that any two pointels of $\DBh{h}{X}$ at
  distance less than $\min(\beta \sqrt{h},2\rho)$ are $k$-visible.
  Furthermore the average distance between $k$-visible pointels of
  $\DBh{h}{X}$ is greater than $\Omega(\sqrt{h})$ for small enough
  $h$. 
\end{theorem}
\begin{proof}
  Two pointels $p$ and $q$ are $k$-visible in $\DBh{h}{X}$ iff the
  segment $\lbrack p q \rbrack$ lies within the $k$-star of
  $\DBh{h}{X}$. Recalling that the realization in $\R^d$ of
  $\kStar{\DBh{h}{X}}{k}$ is exactly $\CBh{h}{X} \oplus k\rbrack -h, h
  \lbrack^d$ ($\oplus$ is the Minkowski sum), a sufficient condition
  is that
  \begin{align*}
    \max_{r \in \lbrack p q \rbrack} d(r,\DBh{h}{X}) \Le kh.
  \end{align*}
  Now let us fix $\beta:=\sqrt{\alpha}-\sqrt{dh}$. From $h <
  \frac{\alpha}{d}$, we have $\beta > 0$. If $\|\vec{pq}\| \Le \beta
  \sqrt{h}$, then $\|\vec{pq}\| \Le \alpha \sqrt{h} - \sqrt{d}h$.
  \RefTheorem{thm-distance-pq-to-dX} then implies
  \begin{align*}
    \max_{r \in \lbrack p, q \rbrack} d( r, \CBh{h}{X}) & \Le (\sqrt{d}+\frac{\alpha}{\
      4\rho})h \\
    & \Le \frac{4\rho\sqrt{d}+4\rho(k-\sqrt{d})}{4\rho}h\\
    & = kh,
  \end{align*}
  so $q$ is $k$-visible from $p$ in $\DBh{h}{X}$.

  We study now the average distance between $k$-visible pointels (in
  $\DBh{h}{X}$).  Let $p$ be any pointel of $\DBh{h}{X}$.  Let $V(p)$
  by the $k$-visible pointels  from $p$, whose cardinal
  is denoted by $N(p)$. We subdivide $V(p)$ into three disjoint sets:
  $V_1(p)$ contains the $k$-visible pointels at distance less than
  $\beta \sqrt{h}/2$ (a non-empty set), $V_2(p)$ the ones at distance between $\beta
  \sqrt{h}/2$  and $\beta \sqrt{h}$ (a non-empty
  set), and $V_3(p)$ the remaining ones (which can be empty). Their
  respective cardinals are denoted by $N_i(p)$, $i\in \{1,2,3\}$.

  The average distance $D(p)$ of $k$-visible points from $p$ can be computed as:
  \begin{align*}
    D(p) & = \frac{1}{N(p)} \sum_{q \in V(p)} \|\vec{pq}\| \\
    & =\frac{1}{N(p)}  \left( \sum_{q \in V_1(p)} \hspace{-2mm}\|\vec{pq}\|
    + \hspace{-2mm} \sum_{q \in V_2(p)} \hspace{-2mm}\|\vec{pq}\| + \hspace{-2mm} \sum_{q \in V_3(p)} \hspace{-2mm} \|\vec{pq}\|
    \right)\\
    & \Ge \frac{1}{N(p)} \left( N_2(p) \frac{\beta\sqrt{h}}{2} + N_3(p) \beta\sqrt{h} \right),
  \end{align*}
  by lower bounding the first term by $0$ and the third term by the
  minimal distance $\beta\sqrt{h}$. Now, since $\partial X$ is smooth
  and looks locally like a plane, the number of pointels of
  $\DBh{h}{X}$ in a ball centered at $p$ with radius $R$ is
  proportional to $R^{d-1} / h^{d-1}$. It coincides with the number of
  $k$-visible pointels from $p$ for $R = \beta \sqrt{h}$, since all
  these pointels are $k$-visible from $p$.  It follows that
  $N_1(p)+N_2(p) \approx c N_2(p)$, with $c=\frac{2^{d}}{2^d-2}$ ($2$ in
  $2$D, $4/3$ in $3$D).  Inserting into the previous inequality gives:
  \begin{align*}
    D(p) & \Ge \frac{1}{c
      N_2(p)+N_3(p)} \left(\frac{N_2(p)}{2}+N_3(p) \right) \beta\sqrt{h}\\
    & \Ge \frac{\beta}{2c}\sqrt{h}.
  \end{align*}
  We can conclude that the average distance $D(p)$ of $k$-visible
  points from $p$ is greater than $\Omega(\sqrt{h})$, since the last
  bound is valid for $c \Ge 1/2$, which is clearly the case for
  arbitrary dimension $d$.
\end{proof}

\textbf{Convergence of visibility normals.} We slightly simplify here
the definition of the $k$-visibility normal $\vec{n}_k(p)$ at point $p$
by removing the weights and by centering the tensor matrix at point
$p$. Let $V_p^k$ be the $k$-visible points in $\DBh{h}{X}$ from $p$,
whose distance does not exceed some $K\sqrt{h}$, for $K$ a chosen
constant ($K=\sqrt{\rho}$ is a reasonnable choice). We have
\begin{align*}
  A(p) & := \frac{1}{|V_p^k|}\sum_{q \in V_p^k} \frac{\vec{pq}}{\| \vec{pq} \|} \otimes \frac{\vec{pq}}{\| \vec{pq} \|},
\end{align*}
and $\vec{n}_k(p)$ is the last eigenvector of $A(p)$ (corresponding to
the largest eigenvalue).
