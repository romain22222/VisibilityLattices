We have proposed a new algorithm that computes the visibility between
all pairs of points on a digital surface. It uses the specificity of a
structure to represent sets of lattice points or cells with a mapping
from projected coordinates to list of integer intervals. Compared to a
classical breadth-first approach to compute visibility, it provides
the exact visibility result, even when the set of visible points from
a source is disconnected. As shown by experiments the running times
are comparable with the breadth-first algorithm, and even faster for
small maximal visibility distance, which are typical when processing
standard 3d images up to $512^3$. We have used visibility to define a
discrete tangent estimator that respects salient features while giving
good approximations on digitization of smooth parts.  This
approximation is even proven to be multigrid convergent on
digitization of smooth surfaces, with a convergence speed of order
$h^{\frac{1}{2}}$ in the worst case. We also explored different
kernels to properly weight the neighborhood data in order to obtain
the best possible normal approximation. The obtained normal vector
field is more suitable to curvature estimation than the classical
Integral Invariant normal estimator.

We have several ideas for possible future works. First, it is possible
to speed up the identification of non visibility by a coarse-to-fine
approach and a precomputation of a pyramid of covering cells. We would
like to explore if we can build such a top-down variant of our
algorithm, considering the primitive vectors as a hierarchy. Moreover,
our algorithm is not limited to pairwise visibility, but is more an
algorithm of exact pattern matching. We would like to explore new
applications of pattern identification along digital surfaces (like
corner detection, locally convex zones identification, etc.). We
finally plan to do edge detection on digital surfaces using the same
algorithm, by identifying patterns on the visibility. On a more
theoretical level, we would like to prove the convergence of the
$1$-visible normal, which is observed in practice. The difficulty lies
in the fact that we can build smooth surface examples with points at
distance $\sqrt{h}$ that are not $1$-visible. However the result seems
to hold on average, so we intend to explore this line of research.
